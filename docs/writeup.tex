\documentclass[12pt,a4paper,draft]{article}

\usepackage[scaled]{helvet}
\renewcommand{\familydefault}{\sfdefault}

\usepackage[T1]{fontenc}
\usepackage[utf8]{inputenc}
\usepackage[draft=false]{hyperref}

\title{Decentralized Systems Engineering~--- Project write-up}
\author{Björn, Gerard and Lukas}
\date{December~12, 2018}

\begin{document}

\maketitle
\tableofcontents

\section{Introduction}

In our group project, we will improve Peerster's file-sharing capabilities by making the service more reliable, more scalable and more secure:

\begin{itemize}
    \item To improve scalability and to distribute the workload among hosts, a file's chunks aren't all hosted by the same node, but rather by multiple nodes.
    This is done by storing the chunks in a distributed hash table when a user want to share a file.
    The hash table uses redundancy to prevent data loss in the case that a node shuts down.
    \item Chunks are transferred and stored by hosts in an encrypted manner to guarantee confidentiality.
    In addition, each chunk gets a cryptographic signature to ensure integrity.
    \item Peerster's blockchain is augmented to store information about the nodes, such as their public keys.
\end{itemize}

\section{Related work}

The central theme at the heart of our project is peer-to-peer file sharing, which has been around for decades and has been explored extensively.
Napster, released in 1999, was one of the services that popularized peer-to-peer file sharing over the Internet.
It used a centralized model, where a central server connected users searching for a file with peers providing the same file.

``If other people have also tried to solve this problem (which they most probably have), what
are the main approaches that they use and what are their drawbacks and limitations?
Try to summarize the main idea of their approaches and describe how your approach compares to
theirs.''
[Stuff related to the project as a whole can go here, and stuff related to a particular part goes in the subsections below.]

[\textasciitilde{}0.5 pages]

\subsection{Part A (Hash-table-related)}

[\textasciitilde{}0.33 pages]

\subsection{Part B (Cryptography-related)}

[\textasciitilde{}0.33 pages]

\subsection{Part C (Blockchain-related)}

Implementing a blockchain with the purpose of having a shared status in which all the nodes can trust is a problem that have been solved several times. The main example when talking about blockchain is Bitcoin. Bitcoin uses a proof of work that consist in calculate a hash SHA-256 starting in a determined number of 0 bits. It also uses a genesis block mined by its creator and has a protocol to request a certain missing block to another node.

[\textasciitilde{}0.33 pages]

\section{System goals, functionalities and architecture}

Goals of our project, functionalities.
Overall architecture, how do the parts work together?
Draw a diagram.
[Stuff related to the project as a whole can go here, and stuff related to a particular part goes in the subsections below.]

[\textasciitilde{}1.5 pages]

\subsection{Part A (Hash-table-related)}

[\textasciitilde{}0.5 pages]

\subsection{Part B (Cryptography-related)}

[\textasciitilde{}0.5 pages]

\subsection{Part C (Blockchain-related)}

In order to be able to share with all the nodes a public key and in order to map it to an unique node, we will implement a blockchain system. 

As we are working in a group of 3 members, the blockchain part will offer an interface to communicate with it, so the others parts of the project doesn't have to understand or deal with the interiorities of it. This interface should support 2 basic functionalities:

\begin{itemize}
 \item Add a public key: Any node has to be able to add its public key to the blockchain. This public key have to be unique in the blockchain.

 \item Get someones public key: In order to decrypt messages and confirm the source of a message the interface will offer the option to get the public key from the node's name.
\end{itemize}

We have already implement a blockchain for homework 3. In that blockchain we had to suppose that all nodes were available from the beginning because we didn't had any way to request the old blocks that had been added to the chain. 
So, if a peer joined to an already ruining network it would had to accept the first block received (because it didn't had any way to validate the block). \\


For this part of our project, we want to add a protocol to be able to request a certain block. With that protocol a peer will be always able to reconstruct the longest chain and, therefore, join later to a network or have a failure and restart.

[\textasciitilde{}0.5 pages]

\end{document}
